\section{Uwagi}
Oba modele układów PLL załączone do sprawozdania zostały zrealizowane z
sukcesem. Oba działają identycznie. Oba wykazują się tym samym błędem po
osiągnięciu synchronizacji. Sygnał generowany przez układ PLL jest przesunięty w
fazie o jeden krok próbkowania do przodu względem sygnału docelowego.

Nie jesteśmy pewni źródła tego błędu. Pomimo licznych prób nie udało nam się go
wyeliminować. Zakładamy, że jest to dowolna kombinacja poniższych elementów:

\subsection{Dokładność kontrolera PI}
\label{sec:uwagi_pi}
Sądzimy, że sygnał wyjściowy kontrolera PI zbiega do nieznacznie większej
wartości niż powinien. Niestety, pomimo licznych prób nie udało nam się dobrać
parametrów dla kontrolera, które niwelowały błąd --- sekcja \ref{sec:pi}.

\subsection{Dokładność generatora sygnału referencyjnego}
Dokładność wartości sygnału referencyjnego ma krytyczne znaczenie dla uzyskania
synchronizacji. Pomimo, wydawałoby się, trywialnej i niezawodnej implementacji
generatora (sekcja \ref{sec:generator}) możliwe jest, że przy aktualnej
kombinacji częstliwości próbkowania $f_{smp} = \unit[10]{kHz}$ i częstotliwości
sygnału referencyjnego $f_{target} = \unit[0.25]{kHz}$ odchodzimy od właściwej
wartości w sąsiedztwie wartości brzegowych $0$ lub $2\pi$.

\subsection{Dokładność filtru dolnoprzepustowego}
Przy aktualnie dobranych parametrach, częstotliwość odcięcia na filtrze wynosi
$f_{cut} = \unit[0,025]{kHz}$. Idealnie byłoby ustalić jeszcze niższą
częstotliwość odcięcia rzędu $\unit[0,001]{kHz}$. Jednak dla narzuconych
parametrów filtru to jest model Butterwortha, rzędu 4. i częstotliwości
próbkowania $f_{smp} = \unit[10]{kHz}$ ustalenie $f_{cut}$ poniżej wartości
$\unit[0,020]{kHz}$ prowadzi do utraty stabilności filtru, a przez to całego
układu. W efekcie jesteśmy zmuszeni ustalić częstotliwość odcięcia na poziomie
$f_{cut} = \unit[0,025]{kHz}$.

W efekcie wartość sygnału wychodzącego z filtru po początkowej
korygacji może być wciąż zbyt wysoka. Idea działania naszego detektora fazy,
zakłada, że filtr dolnoprzepustowy odcina wszystkie sygnały poza sygnałami o
częstotliwościach bliskich $\unit[0]{Hz}$. Poprzez ograniczenie na $f_{cut}$
łamiemy to założenie. Sukcesywnie może to powodować błąd w wartości sygnału
wychodzącego z kontrolera PI, opisany w~\ref{sec:uwagi_pi}.
