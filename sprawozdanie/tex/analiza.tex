\section{Analiza działania}

\subsection{Czas synchronizacji układu PLL}
Zaimplementowany układ PLL synchronizuje się z sygnałem $f_5 = \unit[250]{Hz}$ i
$\varphi_5 = 30\degree$ w czasie krótszym niż $\unit[0,042]{s}$. Jest to sygnał
docelowy podany w treści zadania projektowego z sekcji \ref{sec:zadanie}. Wykres
obrazujący proces synchronizacji jest generowany w MATLABie przez scope o nazwie
\emph{Scope for target and PLL output signals}. Scope wyświetla trzy sygnały:
\begin{itemize}[nosep]
    \item yellow -- sygnał docelowy z zadania,
    \item magenta -- sygnał wyjściowy układu PLL,
    \item cyan -- sygnał referencyjny.
\end{itemize}

\noindent
Parametry operacyjne podzespołów układu zostały dobrane tak, by
uzyskać jak najszybszą synchronizację dla sygnału o fazie $\varphi_5 =
30\degree$. Zmiana przesunięcia fazy $\varphi_5$ powoduje zmianę czasu
synchronizacji, który jest niekoniecznie optymalny.
\begin{table}[h]
    \centering
    \begin{tabular}{cc}
        \toprule
        \textbf{Wartość $\varphi_5$ [$\unit{\degree}$]} & \textbf{Przybliżony czas synchronizacji [$\unit{s}$]} \\
        \midrule
        $5$   & $0,033$ \\
        $10$  & $0,035$ \\
        $15$  & $0,038$ \\
        $30$  & $0,042$ \\
        $45$  & $0,046$ \\
        $60$  & $0,050$ \\
        $90$  & $0,064$ \\
        $120$ & $0,084$ \\
        $180$ & $0,148$ \\
        \bottomrule
    \end{tabular}
    \caption{Przedstawienie zależności czasu synchronizacji od przesunięcia fazowego sygnału docelowego.}
    \label{tab:analiza_czasy}
\end{table}

\subsection{Stabilność układu PLL}
Układ jest stabilny dla dobranych parametrów operacyjnych podzespołów opisanych
w sekcji \ref{sec:wykonanie}. Wartości te zostały dobrane specyficznie dla
przypadku z zadania projektowego. Zwiększenie współczynnika członu
integracyjnego kontrolera PI szybko prowadzi do utraty stabilności całego
układu. Obniżenie częstotliwości odcięcia filtru dolnoprzepustowego ma
identyczny efekt. Dobrane przez nas parametry operacyjne podzespołów są unikalne
dla częstotliwości sygnału, do którego układ PLL ma się synchronizować.
