\section{Treść zadania}
\label{sec:zadanie}

W ramach projektu mieliśmy zamodelować w środowisku MATLAB układ pętli
synchronizacji fali. Układ miał na pojedynczym wejściu przyjmować sumę trzech
sygnałów:
\begin{itemize}[nosep]
    \item $f_1 = \unit[50]{Hz}$ i przesunięciu fazowym $\varphi_1 = 5\degree$,
    \item $f_3 = \unit[150]{Hz}$ i przesunięciu fazowym $\varphi_3 = 10\degree$,
    \item $f_5 = \unit[250]{Hz}$ i przesunięciu fazowym $\varphi_5 = 30\degree$.
\end{itemize}
W otrzymanym sygnale układ miał zsynchronizować się do sygnału o częstotliwości
$f_5$ i podać go na wyjście.
